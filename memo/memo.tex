\documentclass[landscape]{slides}
\usepackage[reals]{layout} %provide \layout macro to show current page layout

\usepackage{cite}

\usepackage{listings}
\lstset{ 
  language=C++,
  morekeywords={Float_t, Int_t, Double_t, Bool_t},
  frame=tbrl,
  frameround=tttt,
  showspaces=false,
  showtabs=false,
  basicstyle=\footnotesize,
% backgroundcolor=\color{Gray},
% fillcolor=\color{Gray},
  extendedchars=true
}            
\lstloadlanguages{sh,bash,csh,[GNU]C++,[gnu]make,SQL}

\usepackage{graphicx}
\usepackage{wrapfig}

\usepackage{amsmath}            % more evironment
\usepackage{amssymb}            % more symbol

\usepackage{ifpdf}
\ifpdf
\usepackage{epstopdf} % must be put after graphicx
\usepackage[usenames,dvipsnames]{color}
\usepackage[pdftex,bookmarks=true]{hyperref}
\pdfadjustspacing=1 %force pdfLaTeX to use the same spacing as LaTeX
\else
\usepackage[usenames,dvips]{color}
%\usepackage[ps2pdf]{hyperref} % cannot be used together with slides
\fi

% Alter some LaTeX defaults for better treatment of floats: See p.105 % of "TeX Unbound" for suggested values. See pp. 199-200 of Lamport's % "LaTeX" book for details.

% General parameters, for ALL pages:
\renewcommand{\topfraction}{0.9} % max fraction of floats at top
\renewcommand{\bottomfraction}{0.9} % max fraction of floats at bottom

% Parameters for TEXT pages (not float pages):
\setcounter{topnumber}{2}
\setcounter{bottomnumber}{2}
\setcounter{totalnumber}{4} % 2 may work better
\renewcommand{\textfraction}{0.1} % allow minimal text w. figs

% Parameters for FLOAT pages (not text pages):
\renewcommand{\floatpagefraction}{0.7}	% require fuller float pages
% N.B.: floatpagefraction MUST be less than topfraction !!

% remember to use [htp] or [htpb] for placement
%\pagestyle{headings}

%% The lineno packages adds line numbers. Start line numbering with
%% \begin{linenumbers}, end it with \end{linenumbers}. Or switch it on
%% for the whole article with \linenumbers.
\usepackage{lineno}

%%% Local Variables: 
%%% mode: latex
%%% TeX-master: "memo"
%%% End: 


\begin{document}
\begin{slide}
\begin{center}
\textcolor{blue}{Memo of DEP events PS analysis}\\
Jing Liu\footnote{mailto:jingliu@mppmu.mpg.de},
Annika Vauth\footnote{mailto:vauth@mppmu.mpg.de}

\end{center}

\end{slide}

% ------------

\begin{slide}

\textbf{Simulation:}

\begin{itemize}

\item Simulated gamma with 2.6 MeV originating 10 cm above vacuum can

\item Background sample: simulated gamma line with 1620.5 keV (Bi-212)

\item Selected single segment events

\item Simulated the pulse shape of the selected events

\end{itemize}

\end{slide}

\begin{slide}

\textbf{Experiment Data}

\begin{itemize}

\item Used Siegfried I data from run with Th228 source on top
	\\ (/.hb/raidg01/Siegfried\_I/Run\_01/Th228\_pulseshape\_core)

\item Gamma line @ 2614.5 keV , \\
	\\ Background Bi-212 gamma line @ 1620.5 keV.

\end{itemize}

\end{slide}

\begin{slide}

\textbf{Signal Data Sample:}

\begin{itemize}

\item Selected the DEP events: single segment events,
	fitted gauss peak, with mean = 1592.7 and sigma = 2.36 keV, 
	\\ made cut for 1589.36 keV $<$ core Energy $<$ 1594.09 keV
	\\ (gives 892 events from data)

\end{itemize}

\begin{center}
\includegraphics[width=0.6\textheight]{eps-images/DEPsingle.eps}
\end{center}

\end{slide}

\begin{slide}

\textbf{Background Data Sample:}

\begin{itemize}

\item Bi-212 gamma line (1620.5 keV), select single segment events \\
	fitt gauss peak, mean = 1620.3 and sigma = 2.45 keV, 
	\\ made cut for 1617.87 keV $<$ core Energy $<$ 1622.78 keV
	\\ (gives 595 events from data)

\end{itemize}

\begin{center}
\includegraphics[width=0.6\textheight]{eps-images/BGsingle.eps}
\end{center}

\end{slide}

\begin{slide}

\textbf{Neural Network:}

\begin{itemize}

\item First prepare file with a training tree \\
     (core waveform info from DEP and background, add ``type'' branch (``signal or background'').

\item Setup a network (TMultiLayerPerceptron)  \\
      waveform-bins as input, hidden neurons, type as output.

\item Train the network. Tried different learning methods: \\
  kBatch, kDescent, kRibierePolak, kFletcherReeves (a lot slower), 
  \\ kStochstic (works sometimes ?!), kBFGS (works sometimes..?!)

\end{itemize}

\end{slide}

%\begin{slide}
%\includegraphics[width=0.6\textwidth,angle=270]{eps-images/NNtraining.eps}
%\end{slide}

%\begin{slide}
%\includegraphics[width=0.6\textwidth,angle=270]{eps-images/NNresult.eps}
%\end{slide}

%\begin{slide}
%\includegraphics[width=0.6\textwidth,angle=270]{eps-images/NNmlp.eps}
%\end{slide}


\end{document}

%%% Local Variables:
%%% mode: latex
%%% TeX-master: t
%%% End: 
